%% fonty
\usepackage{fontspec}
%\setmainfont[Ligatures=TeX]{Stempel Garamond}
\defaultfontfeatures{Mapping=tex-text}
\usepackage{textcase}
\usepackage{booktabs}
\usepackage{multicol}
\sloppy
\renewcommand{\textsc}[1]{\MakeTextUppercase{#1}} % příjmení autorů velkými písmeny

%% typografická zlepšení
%\usepackage{csquotes}

% barvy
\usepackage{xcolor}
\definecolor{mygreen}{cmyk}{0.36,0.0,0.47,0.40}

\usepackage[
   bookmarks=true,
   unicode=true,
   colorlinks=false,
   linkbordercolor=mygreen,% 
   citebordercolor=mygreen,%   
   urlbordercolor=mygreen%
   ]{hyperref}


% nadpisy
\usepackage{titlesec}
\titleformat{\section}
{\Large\bfseries}
{\thesection}{.5em}{}[\titleline{\color{mygreen}\titlerule[3pt]}] 

\titleformat{\subsection}
{\large\bfseries}
{\thesubsection}{.5em}{}[\titleline{\color{mygreen}\titlerule[1pt]}] 

%% floats get barriers
\usepackage[section]{placeins}

%% poznámky pod čarou
\usepackage{footnote} % možnost používat poznámky pod čarou v tabulkách
\makesavenoteenv{tabular} % automaticky se budou ukládat poznámky pod čarou v prostředí tabular
\makesavenoteenv{table} % automaticky se budou ukládat poznámky pod čarou v prostředí table
% zůstává problém, že při kliknutí na odkaz na poznámku pod čarou vytvořený pomocí hyperref nefunguje

% hlavičky
\usepackage{fancyhdr}

%\renewcommand{\chaptermark}[1]{\rightmark{\thechapter\ #1}}
%\renewcommand{\sectionmark}[1]{\rightmark{\thesection\ #1}}

% Nastaví styl záhlaví pro sudé i liché stránky
\fancypagestyle{myfancy}{
  \fancyhf{} % smaže aktuální nastavení záhlaví a zápatí
  \fancyfoot[LE]{\bfseries\thepage} 
  \fancyfoot[RO]{\bfseries\thepage} 
  %\fancyhead[LO]{\itshape\rightmark}
  %\fancyhead[RE]{\itshape\leftmark}

  \renewcommand{\headrulewidth}{0pt} % tloušťka linky
  \renewcommand{\footrulewidth}{0pt}   % patička chybí
  \addtolength{\headheight}{1.2pt} % prostor pro záhlaví
}

\fancypagestyle{plain}{
  \fancyhead{} % na prázdných stránkách nechci záhlaví
  \renewcommand{\headrulewidth}{0pt} % ani linku
}

\newcommand{\frontmatter}{%
  \thispagestyle{empty}%
  \pagestyle{empty}%
}

\newcommand{\mainmatter}{%
  \cleardoublepage\pagenumbering{arabic}%
  \pagestyle{myfancy}%
}



%\let\oldurl\url
%\renewcommand{\url}[1]{<\oldurl{#1}>}

%% citování
\usepackage[backend=biber,
%style=footnote-dw,%authortitle-dw
style=iso-numeric,
%namefont=smallcaps,
%isbn=true,
language=czech,
sortlocale=cs_CZ,
bibencoding=UTF8,
sorting=nyt,
%autocite=footnote,
backref=true,
firstinits=true,
%nopublisher=false,
hyperref]{biblatex} 
%\urlstyle{rm}


% custom - from short
\let\finalandcomma=\!
\renewcommand*{\multinamedelim}{\addcomma\space}
\renewcommand*{\finalnamedelim}{%
\ifnum\value{liststop}>2 \finalandcomma\fi%
\addsemicolon\space}
\renewcommand*{\labelnamepunct}{\addperiod\space}
\renewcommand*{\nametitledelim}{\addperiod\space}
\renewcommand*{\newunitpunct}{\addperiod\space}
% custom


%\DeclareFieldFormat{title}{\mkbibemph{#1}} % titul kurzívou
%\let\cite\autocite

\usepackage[section,nottoc]{tocbibind}% list automatically lof in toc 
\renewcommand{\contentsname}{Seznam použité literatury}
\renewcommand{\refname}{Seznam použité literatury}
\DefineBibliographyStrings{czech}{%
  references = {Seznam použité literatury},
}
%\defbibheading{bibliography}{\section{Seznam použité literatury}} % přejmenování sekce - pouze pokud je členěná bibliografie
\bibliography{../literatura/hlasivky}

%% závěrečné úkony
\AtEndDocument{
\addcontentsline{toc}{section}{Seznam použité literatury}
\printbibliography} % tisk bibliografie na konci souboru

% popiseky tabulek
\usepackage[hang,bf,small]{caption} % úprava popisku tabulky
\setlength{\captionmargin}{20pt}

% tisk částí
\let\stdsection\section
\renewcommand*{\section}{\clearpage\stdsection}
